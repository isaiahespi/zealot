\documentclass[11pt]{article}

\usepackage[]{mathpazo}

\usepackage{amssymb,amsmath}
\usepackage{ifxetex,ifluatex}
\usepackage{fixltx2e} % provides \textsubscript
\ifnum 0\ifxetex 1\fi\ifluatex 1\fi=0 % if pdftex
  \usepackage[T1]{fontenc}
  \usepackage[utf8]{inputenc}
\else % if luatex or xelatex
  \ifxetex
    \usepackage{mathspec}
    \usepackage{xltxtra, xunicode}
  \else
    \usepackage{fontspec}
  \fi
  \defaultfontfeatures{Mapping=tex-text, Scale=MatchLowercase}
  \newcommand{\euro}{€}
\fi

% use upquote if available, for straight quotes in verbatim environments
\IfFileExists{upquote.sty}{\usepackage{upquote}}{}
% use microtype if available
\IfFileExists{microtype.sty}{%
\usepackage{microtype}
\UseMicrotypeSet[protrusion]{basicmath} % disable protrusion for tt fonts
}{}
\usepackage[margin=1in]{geometry}


\makeatletter
\@ifpackageloaded{hyperref}{}{%
\ifxetex
  \usepackage[setpagesize=false, % page size defined by xetex
              unicode=false, % unicode breaks when used with xetex
              xetex]{hyperref}
\else
  \usepackage[unicode=true]{hyperref}
\fi
}
\@ifpackageloaded{color}{
    \PassOptionsToPackage{usenames,dvipsnames}{color}
}{%
    \usepackage[usenames,dvipsnames]{color}
}
\makeatother
\hypersetup{breaklinks=true,
            bookmarks=true,
            pdfauthor={ ()},
            pdftitle={\#\#\#\#: (Course Number)},
            colorlinks=true,
            citecolor=blue,
            urlcolor=blue,
            linkcolor=blue,
            pdfborder={0 0 0}}
\urlstyle{same}  % don't use monospace font for urls
\usepackage{longtable,booktabs}
% \setlength{\parindent}{0pt}
\setlength{\parskip}{6pt plus 2pt minus 1pt}
\setlength{\emergencystretch}{3em}  % prevent overfull lines
\setcounter{secnumdepth}{0}

%%% Use protect on footnotes to avoid problems with footnotes in titles
\let\rmarkdownfootnote\footnote%
\def\footnote{\protect\rmarkdownfootnote}

%%% Change title format to be more compact
\usepackage{titling}

% % Create subtitle command for use in maketitle
% \newcommand{\subtitle}[1]{
%   \posttitle{\large}
%}

%% --------- above is almost identical with default rmarkdown-------------------


\setlength{\droptitle}{-5\baselineskip} % Move the title up
% alternatively
% \setlength{\droptitle}{-2em}

  \title{\bfseries{\color{BrickRed}{Course Title:}} \#\#\#\#}

  \author{\bfseries{Professor:} \#\#\#\#}
  % \preauthor{\centering\large\emph}
  % \postauthor{\par}

  % \predate{\centering\large\emph}
  % \postdate{\par}
  \date{\bfseries{Course Dates:} yyyy-dd-mm to yyyy-mm-dd}

%% --------- above is all the things for the syllabus --------------------------

%% document formatting
 % colors for tables and text
\usepackage{ragged2e} % justifying text
\usepackage{setspace} % spacing commands, automatically makes captions single-spaced
  \setstretch{1.2} \frenchspacing
\usepackage{lastpage} % access number of last page for numbering in margin

%% fonts setup
\usepackage{gensymb} %degree symbol
\usepackage{array}
\usepackage{multirow}
\usepackage{wrapfig}
\usepackage{float}
\usepackage{colortbl}
\usepackage{pdflscape}
\usepackage{tabu}
\usepackage{threeparttable}
\usepackage{threeparttablex}
\usepackage{makecell}
\usepackage[hang]{footmisc}

%% tables and figures


\robustify\tnote

% table/figure captions (titles and legends)

% \usepackage{caption}
% % otherwise just
% \captionsetup{font=small, labelfont={sc, bf}}
% 
% table/figure captions (titles and legends)
\usepackage[singlelinecheck=false]{caption} % left justify table captions
\captionsetup{font={color=black, sf}, skip=4pt, size=small, labelfont=sf, labelsep = period} % color and spacing between caption and table proper, updated

\makeatletter
\let\runtitle\@title
\makeatother



\usepackage{fancyhdr}
\pagestyle{fancy}
\fancyhf{}
\renewcommand{\headrulewidth}{0pt}
\rhead{\footnotesize \nouppercase{\sc \#\#\#\#}}

   
      \renewcommand{\footrulewidth}{0pt}
   \rfoot{\footnotesize \thepage}
   
\fancypagestyle{firststyle}
{
   \fancyhf{}
      \renewcommand{\footrulewidth}{0pt}
   }

\fancypagestyle{nofooter}
{
   \fancyfoot{}
   \renewcommand{\footrulewidth}{0.0pt}
}

\setlength\parindent{0pt}

\usepackage{titlesec}
\titleformat*{\section}{\sc \bf \large \color[rgb]{0,0,0}}
\titleformat*{\subsection}{\bf \large}
\titleformat*{\subsubsection}{\bf \normalsize}
\titlespacing\section{0pt}{8pt plus 2pt minus 2pt}{0pt plus 2pt minus 2pt}
\titlespacing\subsection{0pt}{6pt plus 2pt minus 2pt}{-2pt plus 2pt minus 2pt}
\titlespacing\subsubsection{0pt}{2pt plus 2pt minus 2pt}{-3pt plus 2pt minus 2pt}


%----- set up title page -------------------------------------------------------

\pretitle{
\begin{flushleft}
\includegraphics[height=2.2cm, keepaspectratio]{umd_logo.png}
\end{flushleft}
\begin{flushleft}\LARGE
}

\posttitle{
  \end{flushleft}
  \begin{flushleft}\large \vskip -1.5em (Course Number) \end{flushleft}
}

\preauthor{\par\begin{flushleft}\large\vskip -1em}
\postauthor{\end{flushleft}}

\predate{\par\begin{flushleft}\large\vskip -1em}
\postdate{\end{flushleft}  }



\makeatletter
\def\@seccntformat#1{\llap{\csname the#1\endcsname \hspace{.075 in}}}
\makeatother

% suppress chapter in section numbering
\renewcommand*\thesection{\arabic{section}}

% spacing of bulleted lists
\usepackage{enumitem} %control spacing of enumerated items
\setlist[itemize]{noitemsep, topsep=0pt}

%% remove 'abstract' title, justify text
\renewcommand{\abstractname}{\large}
\renewenvironment{abstract} {\abstractname \justifying \rm \normalsize }

\newlength\tbspace
\setlength\tbspace{.25in}

% units.
%\usepackage{units}


\def\tightlist{}



% Redefines (sub)paragraphs to behave more like sections
\ifx\paragraph\undefined\else
\let\oldparagraph\paragraph
\renewcommand{\paragraph}[1]{\oldparagraph{#1}\mbox{}}
\fi
\ifx\subparagraph\undefined\else
\let\oldsubparagraph\subparagraph
\renewcommand{\subparagraph}[1]{\oldsubparagraph{#1}\mbox{}}
\fi


\usepackage{tikz}

\newcommand{\shrug}[1][]{%
\begin{tikzpicture}[baseline,x=0.8\ht\strutbox,y=0.8\ht\strutbox,line width=0.125ex,#1]
\def\arm{(-2.5,0.95) to (-2,0.95) (-1.9,1) to (-1.5,0) (-1.35,0) to (-0.8,0)};
\draw \arm;
\draw[xscale=-1] \arm;
\def\headpart{(0.6,0) arc[start angle=-40, end angle=40,x radius=0.6,y radius=0.8]};
\draw \headpart;
\draw[xscale=-1] \headpart;
\def\eye{(-0.075,0.15) .. controls (0.02,0) .. (0.075,-0.15)};
\draw[shift={(-0.3,0.8)}] \eye;
\draw[shift={(0,0.85)}] \eye;
% draw mouth
\draw (-0.1,0.2) to [out=15,in=-100] (0.4,0.95);
\end{tikzpicture}}


\usepackage{booktabs}
\usepackage{longtable}
\usepackage{array}
\usepackage{multirow}
\usepackage{wrapfig}
\usepackage{float}
\usepackage{colortbl}
\usepackage{pdflscape}
\usepackage{tabu}
\usepackage{threeparttable}
\usepackage{threeparttablex}
\usepackage[normalem]{ulem}
\usepackage{makecell}
\usepackage{xcolor}

%-------------------------------------------------------------------------------
\begin{document}

\maketitle


\noindent \begin{tabular*}{\textwidth}{ @{\extracolsep{\fill}} lr @{\extracolsep{\fill}}}

\textbf{Term:} Season, Year \\
\textbf{Credits:} \shrug & \textbf{Teaching Assistant:} \shrug \\
\textbf{E-mail:} \texttt{\href{mailto:your.email@address.edu}{\nolinkurl{your.email@address.edu}}} & \textbf{Teaching Assistant E-Mail:} \shrug \\
\textbf{Class Hours:} \shrug & \textbf{Class Room:} \shrug \\
\textbf{Office:} \shrug & \textbf{Office Hours:} \shrug \\
	\end{tabular*}

\vspace{2mm}





\hypertarget{course-description}{%
\section{Course Description}\label{course-description}}

Consider a brief description of the course to give students a broad
sense of what the course is all about. Research indicates that writing
in a welcoming tone can help students engage with a syllabus more
deeply. You might provide some context for the course that helps
students understand why they would take it and how it fits into a
particular sequence or major requirement (if applicable). You could also
explain how the course relates to future career paths, and whether there
are any prerequisites for the course.

\hypertarget{learning-outcomes}{%
\subsection{Learning Outcomes}\label{learning-outcomes}}

After successfully completing this course you will be able to:

\begin{itemize}
\tightlist
\item
  State three to five outcomes that you expect your students will
  achieve as a result of this class.
\item
  Outcomes should be clear and measurable and worded around what
  students do (e.g., define, contrast, apply, analyze, create,
  evaluate).
\item
  If your course carries a
  \href{https://gened.umd.edu/faculty/general-education-learning-outcomes-and-assessment-rubrics}{General
  Education designation, review the learning outcomes here}
\end{itemize}

\hypertarget{required-resources}{%
\subsection{Required Resources}\label{required-resources}}

\begin{itemize}
\tightlist
\item
  Course Website: \url{elms.umd.edu}
\item
  Book: (Indicate that the book is required or recommended. Include ISBN
  number and estimated costs)
\item
  Application/Software (Indicate that the product is required or
  recommended. Include the product link and estimated costs). If
  required and if the application/software has NOT been reviewed by the
  UMD IT to assess security compliance (accessibility and privacy),
  please provide an alternative solution for the students to equally
  succeed and meet the course requirements.
\item
  Total Estimated costs of required course materials: \(\$XX.XX\)
\end{itemize}

\hypertarget{course-structure}{%
\subsection{Course Structure}\label{course-structure}}

Explain the structure of the course here including elements of how the
work outside and inside the class should be balanced.

Example: This course has 4 live-sessions via WebEx that are mandatory.
The flexible framework does not require you to be in a specific location
to participate; however, you must have access to a full-screen computer
or tablet for each live session. (If you use a tablet for the live
session, you must be comfortable typing responses on it.) The online
nature of this class will push you to take an active role in the
learning process. You will do this by engaging and collaborating with
other students and the instructor on a regular basis both, in live
sessions, as well as through group work and activities.

{[}If the course is online, you may wish to include this section{]}

\hypertarget{tips-for-success-in-an-online-course}{%
\subsection{Tips for Success in an Online
Course}\label{tips-for-success-in-an-online-course}}

\begin{enumerate}
\def\labelenumi{\arabic{enumi}.}
\tightlist
\item
  Participate. Discussions and group work are a critical part of the
  course, and I invite you to engage deeply, ask questions, and talk
  about the course content with your classmates. You can learn a great
  deal from discussing ideas and perspectives with your peers and
  professor. Participation can also help you articulate your thoughts
  and develop critical thinking skills.
\item
  Manage your time. Students are often very busy, and I understand that
  you have obligations outside of this class. However, it's important
  that I note that students do best when they have adequate time to
  devote to the class. Schedule time for your online learning and
  participation in discussions each week. Give yourself plenty of time
  to complete assignments including extra time to handle any technology
  related problems.
\item
  Login regularly. I recommend that you log in to ELMS-Canvas several
  times a week to view announcements, discussion posts and replies to
  your posts. You may need to log in multiple times a day when group
  submissions are due.
\item
  Do not fall behind. This class moves at a quick pace and each week
  builds on the previous content. If you feel you are starting to fall
  behind, check in with the instructor as soon as possible so we can
  troubleshoot together. It will be hard to keep up with the course
  content if you fall behind in the pre-work or post-work.
\item
  Use ELMS-Canvas notification settings. Pro tip! Canvas ELMS-Canvas can
  ensure you receive timely notifications in your email or via text. Be
  sure to enable announcements to be sent instantly or daily.
\item
  Ask for help if needed. If you need help with ELMS-Canvas or other
  technology, IT Support. If you are struggling with a course concept,
  reach out to me and your classmates for support.
\end{enumerate}

\hypertarget{policies-and-resources-for-undergraduate-courses-delete-if-you-are-teaching-a-graduate-course}{%
\subsection{Policies and Resources for Undergraduate Courses
\textasciitilde(delete if you are teaching a graduate
course)}\label{policies-and-resources-for-undergraduate-courses-delete-if-you-are-teaching-a-graduate-course}}

It is our shared responsibility to know and abide by the University of
Maryland's policies that relate to all courses, which include topics
like: - Academic integrity - Student and instructor conduct -
Accessibility and accommodations - Attendance and excused absences -
Grades and appeals - Copyright and intellectual property

Please visit \href{www.ugst.umd.edu/courserelatedpolicies.html}{the
Office of Undergraduate Studies} for the full list of campus-wide
policies and follow up with me if you have questions.

\hypertarget{policies-and-resources-for-graduate-courses-delete-if-you-are-teaching-an-undergraduate-course}{%
\subsection{Policies and Resources for Graduate Courses
\textasciitilde(delete if you are teaching an undergraduate
course)}\label{policies-and-resources-for-graduate-courses-delete-if-you-are-teaching-an-undergraduate-course}}

It is our shared responsibility to know and abide by the University of
Maryland's policies that relate to all courses, which include topics
like: - Academic integrity - Student and instructor conduct -
Accessibility and accommodations - Attendance and excused absences -
Grades and appeals - Copyright and intellectual property Please see the
\href{https://gradschool.umd.edu/course-related-policies}{University's
website for graduate course-related policies}

\hypertarget{course-guidelines}{%
\section{Course Guidelines}\label{course-guidelines}}

\hypertarget{namespronouns-and-self-identifications}{%
\subsection{Names/Pronouns and
Self-Identifications:}\label{namespronouns-and-self-identifications}}

The University of Maryland recognizes the importance of a diverse
student body, and we are committed to fostering inclusive and equitable
classroom environments. I invite you, if you wish, to tell us how you
want to be referred to in this class, both in terms of your name and
your pronouns (he/him, she/her, they/them, etc.). Keep in mind that the
pronouns someone uses are not necessarily indicative of their gender
identity. Visit \url{trans.umd.edu} to learn more.

Additionally, it is your choice whether to disclose how you identify in
terms of your gender, race, class, sexuality, religion, and dis/ability,
among all aspects of your identity (e.g., should it come up in classroom
conversation about our experiences and perspectives) and should be
self-identified, not presumed or imposed. I will do my best to address
and refer to all students accordingly, and I ask you to do the same for
all of your fellow Terps.

\hypertarget{communication-with-instructor}{%
\subsection{Communication with
Instructor:}\label{communication-with-instructor}}

\textbf{Email}: If you need to reach out and communicate with me, please
email me at \url{your.email@umd.edu}. Please DO NOT email me with
questions that are easily found in the syllabus or on ELMS (i.e.~When is
this assignment due? How much is it worth? etc.) but please DO reach out
about personal, academic, and intellectual concerns/questions. While I
will do my best to respond to emails within 24 hours, you will more
likely receive email responses from me on Mondays, Wednesdays and
Fridays from 8:00am-10:00am EST

\textbf{ELMS}: I will send IMPORTANT announcements via ELMS messaging.
You must make sure that your email \& announcement notifications
(including changes in assignments and/or due dates) are enabled in ELMS
so you do not miss any messages. You are responsible for checking your
email and Canvas/ELMS inbox with regular frequency.

\hypertarget{communication-with-peers}{%
\subsection{Communication with Peers:}\label{communication-with-peers}}

With a diversity of perspectives and experience, we may find ourselves
in disagreement and/or debate with one another. As such, it is important
that we agree to conduct ourselves in a professional manner and that we
work together to foster and preserve a virtual classroom environment in
which we can respectfully discuss and deliberate controversial
questions. I encourage you to confidently exercise your right to free
speech---bearing in mind, of course, that you will be expected to craft
and defend arguments that support your position. Keep in mind, that free
speech has its limit and this course is NOT the space for hate speech,
harassment, and derogatory language. I will make every reasonable
attempt to create an atmosphere in which each student feels comfortable
voicing their argument without fear of being personally attacked,
mocked, demeaned, or devalued.

Any behavior (including harassment, sexual harassment, and racially
and/or culturally derogatory language) that threatens this atmosphere
will not be tolerated. Please alert me immediately if you feel
threatened, dismissed, or silenced at any point during our semester
together and/or if your engagement in discussion has been in some way
hindered by the learning environment.

\hypertarget{major-assignments}{%
\section{Major Assignments}\label{major-assignments}}

\hypertarget{homework-assignments}{%
\subsection{Homework Assignments}\label{homework-assignments}}

\begin{itemize}
\tightlist
\item
  Reading reflections, video questions and other assignments
\item
  What is their purpose? How will they be conducted?
\end{itemize}

\hypertarget{quizzes-weekly-summaries}{%
\subsection{Quizzes \& Weekly
Summaries}\label{quizzes-weekly-summaries}}

\begin{itemize}
\tightlist
\item
  How many of these will there be
\item
  What is their purpose? How will they be conducted?
\end{itemize}

\hypertarget{participation-engagement}{%
\subsection{Participation \&
Engagement}\label{participation-engagement}}

\begin{itemize}
\tightlist
\item
  During live sessions
\item
  During group discussion boards
\end{itemize}

\hypertarget{team-project}{%
\subsection{Team Project}\label{team-project}}

\begin{itemize}
\tightlist
\item
  What are the components of the team project
\item
  What is the purpose? Where can they go to find more information? ELMS
  link
\end{itemize}

\hypertarget{final-exam}{%
\subsection{Final Exam}\label{final-exam}}

\begin{itemize}
\tightlist
\item
  xxx
\item
  Clarify what tools are needed and where to access study materials
\end{itemize}

\hypertarget{grading-structure}{%
\section{Grading Structure}\label{grading-structure}}

\begin{longtable}[]{@{}ll@{}}
\toprule()
Assignment & Percentage \% \\
\midrule()
\endhead
Homework & 30\% \\
Quizzes \& Weekly Summaries & 15\% \\
Participation/Engagement & 15\% \\
Team Project/Paper/Presentation & 20\% \\
Final Exam & 20\% \\
Total & 100\% \\
\bottomrule()
\end{longtable}

\hypertarget{academic-integrity}{%
\section{Academic Integrity}\label{academic-integrity}}

Describe your expectations for academic integrity. The following is
sample text regarding the use of Turnitin:

For this course, some of your assignments will be collected via Turnitin
on our course ELMS page. I have chosen to use this tool because it can
help you improve your scholarly writing and help me verify the integrity
of student work. For information about Turnitin, how it works, and the
feedback reports you may have access to, visit
\href{https://umd.service-now.com/itsc?id=kb_article\&sys_id=c0116d8f0f7ef2007f232ca8b1050e63}{Turnitin
Originality Checker for Students}

The University's Code of Academic Integrity is designed to ensure that
the principles of academic honesty and integrity are upheld. In
accordance with this code, the University of Maryland does not tolerate
academic dishonesty. Please ensure that you fully understand this code
and its implications because all acts of academic dishonesty will be
dealt with in accordance with the provisions of this code. All students
are expected to adhere to this Code. It is your responsibility to read
it and know what it says, so you can start your professional life on the
right path. \textbf{As future professionals, your commitment to high
ethical standards and honesty begins with your time at the University of
Maryland.} It is important to note that course assistance websites, such
as \texttt{CourseHero}, are not permitted sources, unless the instructor
explicitly gives permission for you to use one of these sites. Material
taken or copied from these sites can be deemed unauthorized material and
a violation of academic integrity. These sites offer information that
might not be accurate and that shortcut the learning process,
particularly the critical thinking steps necessary for college-level
assignments. Additionally, students may naturally choose to use online
forums for course-wide discussions (e.g., Group lists or chats) to
discuss concepts in the course. However, collaboration on graded
assignments is strictly prohibited unless otherwise stated. Examples of
prohibited collaboration include: asking classmates for answers on
quizzes or exams, asking for access codes to clicker polls, etc. Please
visit the
\href{http://www.ugst.umd.edu/courserelatedpolicies.html}{Office of
Undergraduate Studies' full list of campus-wide policies} and reach out
if you have questions.

Finally, on each exam or assignment you must write out and sign the
following pledge:

\begin{quote}
\textbf{``I pledge on my honor that I have not given or received any
unauthorized assistance on this exam/assignment.''}
\end{quote}

If you ever feel pressured to comply with someone else's academic
integrity violation, please reach out to me straight away. Also,
\textbf{if you are ever unclear} about acceptable levels of
collaboration, \textbf{please ask}!

\hypertarget{grades}{%
\section{Grades}\label{grades}}

Campus Policy dictates that you must specify:

\begin{itemize}
\tightlist
\item
  How final letter grades will be determined. This should include a
  breakdown of all graded assessments, their weight in the course, and
  whether final grades will include \(\pm\) descriptors.
\item
  How students will have access to their grades throughout the semester,
  and how they can review their work (including the final exam).
\end{itemize}

All assessment scores will be posted on the course ELMS page. If you
would like to review any of your grades (including the exams), or have
questions about how something was scored, please email me to schedule a
time for us to meet and discuss.

Late work will not be accepted for course credit so please plan to have
it submitted well before the scheduled deadline. I am happy to discuss
any of your grades with you, and if I have made a mistake I will
immediately correct it. Any formal grade disputes must be submitted in
writing and within one week of receiving the grade. Final letter grades
are assigned based on the percentage of total assessment points earned.
To be fair to everyone I have to establish clear standards and apply
them consistently, so please understand that being close to a cutoff is
not the same as making the cut (\(89.99 \neq 90.00\)). It would be
unethical to make exceptions for some and not others.

A table of the assessments and point values can be a concise way to
convey all of the graded elements and their relative weight in the
course. If you are using weighted percentages (e.g., exams = 30\%, paper
= 20\%) be sure to clarify the number of assessments within each
category\ldots{} is there one exam worth 30\% or are there three exams
that are each worth 10.

It is essential that you articulate in your syllabus how final letter
grades will be assigned. There is no campus policy on percentages and
letter grades, nor is there a requirement that you utilize a
points-based scheme.

\hypertarget{resources-accommodations}{%
\section{Resources \& Accommodations}\label{resources-accommodations}}

\hypertarget{accessibility-and-disability-services}{%
\subsection{Accessibility and Disability
Services}\label{accessibility-and-disability-services}}

The University of Maryland is committed to creating and maintaining a
welcoming and inclusive educational, working, and living environment for
people of all abilities. The University of Maryland is also committed to
the principle that no qualified individual with a disability shall, on
the basis of disability, be excluded from participation in or be denied
the benefits of the services, programs, or activities of the University,
or be subjected to discrimination. The
\href{https://www.counseling.umd.edu/ads/}{Accessibility and Disability
Service (ADS)} provides reasonable accommodations to qualified
individuals to provide equal access to services, programs and
activities. ADS cannot assist retroactively, so it is generally best to
request accommodations several weeks before the semester begins or as
soon as a disability becomes known. Any student who needs accommodations
should contact me as soon as possible so that I have sufficient time to
make arrangements.

For assistance in obtaining an accommodation, contact Accessibility and
Disability Service at 301-314-7682, or email them at
\url{adsfrontdesk@umd.edu}. Information about sharing your
accommodations with instructors, note taking assistance and more is
available from the Counseling Center.

\hypertarget{student-resources-and-services}{%
\section{Student Resources and
Services}\label{student-resources-and-services}}

Taking personal responsibility for your own learning means acknowledging
when your performance does not match your goals and doing something
about it. I hope you will come talk to me so that I can help you find
the right approach to success in this course, and I encourage you to
visit \href{http://tutoring.umd.edu}{UMD's Student Academic Support
Services website} to learn more about the wide range of campus resources
available to you.

In particular, everyone can use some help sharpening their communication
skills (and improving their grade) by visiting
\href{http://www.english.umd.edu/academics/writingcenter/schedule}{UMD's
Writing Center} and schedule an appointment with the campus Writing
Center.

You should also know there are a wide range of resources to support you
with whatever you might need
\href{https://sph.umd.edu/content/student-resources-and-services}{UMD's
Student Resources and Services website} may help. If you feel it would
be helpful to have someone to talk to, visit
\href{https://www.counseling.umd.edu/}{UMD's Counseling Center} or
\href{https://tltc.umd.edu/instructors/teaching-topics/supporting-whole-student}{one
of the many other mental health resources on campus}.

\hypertarget{basic-needs-security}{%
\subsection{Basic Needs Security}\label{basic-needs-security}}

If you have difficulty affording groceries or accessing sufficient food
to eat every day, or lack a safe and stable place to live, please visit
\href{https://studentaffairs.umd.edu/basic-needs-security}{UMD's
Division of Student Affairs website} for information about resources the
campus offers you and let me know if I can help in any way.

\hypertarget{veteran-resources}{%
\subsection{Veteran Resources}\label{veteran-resources}}

UMD provides some additional supports to our student veterans. You can
access those resources at the office of
\href{https://stamp.umd.edu/engagement/veteran_student_life}{Veteran
Student life} and the
\href{https://www.counseling.umd.edu/aboutus/}{Counseling Center}
Veterans and active duty military personnel with special circumstances
(e.g., upcoming deployments, drill requirements, disabilities) are
welcome and encouraged to communicate these, in advance if possible, to
the instructor.

\hypertarget{netiquette-policy-optional}{%
\subsection{Netiquette Policy
{[}Optional{]}}\label{netiquette-policy-optional}}

Netiquette is the social code of online classes. Students share a
responsibility for the course's learning environment. Creating a
cohesive online learning community requires learners to support and
assist each other. To craft an open and interactive online learning
environment, communication has to be conducted in a professional and
courteous manner at all times, guided by common sense, collegiality and
basic rules of etiquette.

\hypertarget{participation}{%
\subsubsection{Participation}\label{participation}}

\begin{itemize}
\tightlist
\item
  Given the interactive style of this class, attendance will be crucial
  to note-taking and thus your performance in this class. Attendance is
  particularly important also because class discussion will be a
  critical component for your learning.
\item
  Each student is expected to make substantive contributions to the
  learning experience, and attendance is expected for every session.
\item
  Students with a legitimate reason to miss a live session should
  communicate in advance with the instructor, except in the case of an
  emergency.
\item
  Students who miss a live session are responsible for learning what
  they miss from that session. -- Additionally, students must complete
  all readings and assignments in a timely manner in order to fully
  participate in class.
\end{itemize}

\hypertarget{course-evaluation}{%
\subsection{Course Evaluation}\label{course-evaluation}}

Please submit a course evaluation through Student Feedback on Course
Experiences in order to help faculty and administrators improve teaching
and learning at Maryland. All information submitted to Course
Experiences is confidential. Campus will notify you when Student
Feedback on Course Experiences is open for you to complete your
evaluations at the end of the semester. Please go directly to the
\href{http://courseexp.umd.edu/}{Student Feedback on Course Experiences}
to complete your evaluations. By completing all of your evaluations each
semester, you will have the privilege of accessing through Testudo the
evaluation reports for the thousands of courses for which 70\% or more
students submitted their evaluations.

\hypertarget{copyright-notice}{%
\section{Copyright Notice}\label{copyright-notice}}

Course materials are copyrighted and may not be reproduced for anything
other than personal use without written permission.

\newpage

\hypertarget{course-outline}{%
\section{Course Outline}\label{course-outline}}

The format of this section will vary based on the design of your course
and the semester, but our guidance is to aim for a clear and concise
table that maps out all of the assignment assessments and deadlines and
gives students a sense of the course's organization.

\end{document}
